\documentclass[a4paper,12pt]{report}

\usepackage[utf8]{inputenc}
\usepackage[spanish]{babel}
\usepackage[dvips]{graphicx,color}
\usepackage{calc}
\usepackage{multirow}
\usepackage{colortbl}
\usepackage{xcolor}
\usepackage{moreverb}
\usepackage{hyperref}
\hypersetup{colorlinks=false}
\usepackage[all]{hypcap}

\newcommand{\fig}[3]
{
	\begin{figure}[ht!]
		\centering
		\includegraphics[scale=0.4]{#1}
		\caption{#2.\label{#3}}
	\end{figure}
}

\newcommand{\PICTURE}[4]
{
	\begin{figure}[ht!]\centering\begin{picture}(#1)#2\end{picture}
	\caption{#3.\label{#4}}\end{figure}
}

\newcommand{\TABLE}[4]
{
	\begin{table}[ht!]\centering
	\begin{tabular}{#1}\hline#2\\\hline\end{tabular}
	\caption{#3.\label{#4}}\end{table}
}

\title
{
	{\bf \Large Manual de usuario del programa \emph{SWOCS}}\\
	{\large	versión 1.0}
}
\author
{
	{\bf Autores:} Javier Burguete Tolosa y Samuel Ambroj Pérez\\
	{\small Copyright~\copyright~2011-2012 Javier Burguete Tolosa.
	Todos los derechos reservados}
}
\date{\today}

\hyphenation
{
	an-ti-di-fu-si-vo
	re-sul-ta-dos
}

\newcommand{\swocs}{\emph{SWOCS}}
\newcommand{\IT}[1]{{\sl ``#1''}}

\begin{document}

\maketitle

\tableofcontents

\setlength{\parskip}{\baselineskip / 2}

\chapter{Introducción}

El programa {\swocs} ({\it Shallow Water Open Channel Simulator}) está diseñado para resolver numéricamente flujos
transitorios unidimensionales y transporte con difusión de solutos en un canal
o surco, con presencia de infiltración. 
Existen versiones para sistemas operativos Unix y Windows. Windows es una 
marca registrada de Microsoft Corporation.

El código ha sido escrito en lenguaje C. El programa se distribuye junto con 
el código fuente, siendo de libre uso y distribución bajo una licencia de tipo BSD.  

\section{Internacionalización}

El programa {\swocs} está escrito en inglés. Para evitar problemas con
el formato de los ficheros de entrada y salida con distintas configuraciones
regionales, se ha fijado el formato de los números reales en el estándar
internacional. Por tanto, debe usarse el carácter \IT{.} para señalar la coma
decimal. El carácter \IT{e}, bien en mayúsculas o en minúsculas, indica el
exponente del número. Todas las unidades están expresadas en el Sistema
Métrico Decimal.

\chapter{Instrucciones de instalación y ejecución}

\section{Requisitos de compilación}
Para compilar el programa {\swocs} en sistemas operativos de tipo Unix se
necesita tener instalados:
\begin{verbatimtab}
	gcc, GNU make
\end{verbatimtab}
En Windows puede realizarse la compilación a través de las plataformas
\emph{cygwin} (\url{http://www.cygwin.com/}) y \emph{MinGW}
(\url{http://www.mingw.org/}). 

También pueden usarse otros compiladores. En este caso el archivo de compilación
proporcionado ({\tt makefile}) no es compatible, y sería necesario crear algún
tipo de fichero o proyecto para compilar el conjunto de ficheros.


\section{Instalación en el sistema operativo Unix}

La instalación de este programa en Unix consta de los siguientes pasos:
  \begin{enumerate}
	\item Descomprima el fichero \emph{swocs-src-1-0.zip}. \\
		Al descomprimir el archivo aparecen archivos con extensiones {\tt .c},
		{\tt .h} y un archivo de compilación ({\tt makefile}) en formato
		\emph{GNU make}.
	\item En un terminal ejecute la instrucción: {\tt make} \\
			Este paso crea el ejecutable {\tt swocs} 
	\item También pueden descargarse:
		\begin{itemize}
			\item el manual de usuario en español
				({\it swocs-manual-usuario-1-0.pdf})
			\item el manual de usuario en inglés
				({\it swocs-user-manual-1-0.pdf})
			\item y un manual de referencia del programador\\
				({\it swocs-reference-manual-1-0.pdf}). 
		\end{itemize}
  \end{enumerate}

\section{Instalación en el sistema operativo Microsoft Windows}

Para realizar la instalación en Windows sólo necesita un paso: 
  \begin{enumerate}
	\item Descargue el ejecutable \emph{swocs.exe}. \\
		Si descarga este fichero no necesitará realizar la compilación de los
		códigos fuente. Si desea compilar a partir de los códigos fuente
		descargue y descomprima el fichero \emph{swocs-src-1-0.zip} y proceda
		igual que la compilación en Unix descrita en la sección anterior.
  \end{enumerate}

\section{Ejecución del programa}

En sistemas Microsoft Windows, para ejecutar el programa vaya al modo de comandos:
\begin{description}
\item[Inicio$\rightarrow$Programas$\rightarrow$Accesorios$\rightarrow$Símbolo del sistema]
\end{description}
Después acceda usando la orden \emph{cd} al directorio donde se ubica el ejecutable.
\begin{description}
\item[Ejemplo:] cd C:$\backslash$Programas$\backslash$Swocs
\end{description}

La sintaxis de ejecución, tanto en sistemas Unix como en Microsoft Windows, de una simulación es la siguiente:
\begin{description}
\item \emph{swocs}  \emph{fichero1} \emph{fichero2} \emph{fichero3} \emph{fichero4} \emph{fichero5} \emph{fichero6}
\end{description}
donde los ficheros son:
\begin{description}
\item[\it fichero1,] fichero de entrada del cauce.
\item[\it fichero2,] fichero de resultados de las variables.
\item[\it fichero3,] fichero de resultados de los flujos.
\item[\it fichero4,] fichero de resultados del avance.
\item[\it fichero5,] fichero de entrada de las sondas.
\item[\it fichero6,] fichero de resultados de las sondas.
\end{description}

El formato de cada uno de los 6 ficheros se explica en el capítulo posterior.

\chapter{Formato de los ficheros de entrada y salida}

Todos ficheros de entrada y salida utilizan codificación ASCII, pudiéndose crear y modificar 
con una gran variedad de editores de texto ({\tt notepad}, {\tt vim}).

\section{Formato del fichero de entrada del cauce}
Se muestra en primer lugar un ejemplo de un fichero de entrada del cauce:

{\footnotesize
\begin{boxedverbatim}
100 0 0.14 1.222 0.27
2
1 1 1
0.062
9.6e-06 0.115 0 0.8
11.82
1
0 0.001
4
567 0
567 0.0106
867 0.0106
867 0
401 1
3900 60 0.9 0.01
2 2
1
\end{boxedverbatim}
}

Cada valor tiene la siguiente correspondencia:

{\footnotesize
\begin{boxedverbatim}
(longitud) (pendiente fondo) (anchura fondo) (pendiente pared) (profundidad)
(tipo de salida)
(modelo de fricción) (modelo de infiltración) (modelo de difusión)
(coeficientes de fricción) ...
(coeficientes de infiltración) ...
(coeficientes de difusión) ...
(número de puntos hidrograma de entrada de agua)
(tiempo hidrograma de entrada) (caudal hidrograma de entrada)
...
(número de puntos hidrograma de entrada de soluto)
(tiempo hidrograma entrada soluto) (caudal hidrograma entrada soluto)
...
(número de nodos de la malla) (tipo de condiciones iniciales)
(tiempo final) (intervalo tiempo entre medidas) (número CFL) (altura mínima)
(modelo numérico flujo de superficie) (modelo numérico de la difusión)
(modelo físico de la superficie flujo)
\end{boxedverbatim}
}

\subsection{Tipo de salida}
\begin{verbatimtab}
	1: Cerrada.
	2: Abierta.
\end{verbatimtab}

\subsection{Modelos y coeficientes de fricción}
En la versión actual sólo se encuentra disponible el modelo de fricción de Gauckler-Manning. El único coeficiente de fricción necesario para este modelo es el número de Gauckler-Manning, $n$. Nuevos modelos de fricción podrían necesitar más de un único coeficiente de fricción.
\begin{verbatimtab}[4]
(modelo de fricción): 
		1: Gauckler-Manning.
(coeficientes de fricción):
\end{verbatimtab}
\vspace{-0.60cm}
\hspace{1.8cm}$n$

En el caso de Gauckler-Manning la formulación empleada es la siguiente (\cite{JaviSurcos1}):
\begin{equation}
S_f=\frac{n^2 Q|Q|P^{4/3}}{A^{10/3}}
\end{equation}
\noindent donde $S_f$ es la pendiente de fricción, $Q$ es el caudal superficial, $P$ es el perímetro mojado de la 
sección transversal superficial y $A$ es el área mojada de la sección transversal superficial. 

\subsection{Modelos y coeficientes de infiltración}
En la versión actual sólo se encuentra disponible el modelo de infiltración de Kostiakov-Lewis. Para este modelo son necesarios 4 coeficientes de infiltración. Nuevos modelos de infiltración podrían necesitar un número distinto de coeficientes de infiltración.
\begin{verbatimtab}[4]
(modelo de infiltración): 
		1: Kostiakov-Lewis.
(coeficientes de infiltración):
\end{verbatimtab}
\vspace{-0.5cm}
\hspace{1.8cm}$K$\hspace{0.9cm}$a$\hspace{0.9cm}$i_c$\hspace{0.9cm}$D$

La formulación del modelo de Kostiakov-Lewis (\cite{JaviSurcos1}):
\begin{equation}
\frac{d\alpha}{dt}=P\left[ i_c + Ka \left( \frac{\alpha}{DK} \right)^{a-1/a} \right]
\end{equation}
\noindent donde $\alpha$ es el área infiltrada, $t$ es la coordenada temporal, $i_c$ es la velocidad de infiltración en suelo saturado, $K$ es el parámetro de infiltración de Kostiakov, $a$ es el coeficiente de infiltración de Kostiakov y $D$ es la distancia entre surcos.

\subsection{Modelos y coeficientes de difusión}
En la versión actual sólo se encuentra disponible el modelo de difusión de Rutherford. El único coeficiente de difusión necesario para este modelo
es el coeficiente adimensional de difusión superficial, $D_x$.
\begin{verbatimtab}[4]
(modelo de difusión): 
		1: Rutherford.
(coeficientes de difusión):
\end{verbatimtab}
\vspace{-0.5cm}
\hspace{1.8cm}$D_x$ 

La formulación del modelo de difusión de Rutherford (\cite{JaviSurcos1}):
\begin{equation}
K_x=D_x\sqrt{gAP|S_f|}
\end{equation}
\noindent donde $K_x$ es el coeficiente de difusión superficial y $g$ es la constante gravitatoria.

\subsection{Tipos y parámetros de las condiciones iniciales}
\begin{verbatimtab}[4]
		1: Cauce seco.
\end{verbatimtab}

\subsection{Modelos numéricos del flujo de superficie}
\begin{verbatimtab}
	1: McCormack.
	2: Upwind.
\end{verbatimtab}

\subsection{Modelos numéricos de la difusión}
\begin{verbatimtab}
	1: Explícito.
	2: Implícito.
\end{verbatimtab}

\subsection{Modelos físicos de la superficie de flujo}
\begin{verbatimtab}
	1: Completo.
	2: Cero-inercia.
	3: Difusivo.
	4: Cinemático.
\end{verbatimtab}


\section{Formato del fichero de resultados de las variables}
El fichero de resultados de las variables presenta el valor en la coordenada longitudinal, $x$, de 5 variables de interés, para el tiempo final de la simulación. El orden es el mostrado a continuación:

\hspace{1.8cm}$x$\hspace{0.9cm}$A$\hspace{0.9cm}$Q$\hspace{0.9cm}$A\cdot s$\hspace{0.9cm}$\alpha$\hspace{0.9cm}$\alpha\cdot s_i$ 

\hspace{3.8cm} $\cdots \cdots \cdots$

\noindent donde $s$ es la concentración de soluto promedio en la sección transversal superficial y $s_i$ es la concentración de soluto infiltrada.

\section{Formato del fichero de resultados de los flujos}
El fichero de resultados de los flujos presenta el valor en la coordenada longitudinal, $x$, de 4 variables de interés, para el tiempo final de la simulación. El orden es el mostrado a continuación:

\hspace{1.8cm}$x$\hspace{0.9cm}$\frac{\delta(Qu)}{\delta x}$\hspace{0.9cm}$-gA\frac{\delta z_b}{\delta x}$\hspace{0.9cm}$gA\frac{\delta h}{\delta x}$
\hspace{0.9cm}$gAS_f$ 

\hspace{3.8cm} $\cdots \cdots \cdots$

\noindent donde $u$ es la velocidad promedio en la sección transversal superficial, $h$ es la altura de agua superficial y $z_b$ es la 
coordenada vertical del fondo de la sección transversal. 

\section{Formato del fichero de resultados del avance}

Este fichero guarda para cada paso temporal la pareja de valores:

\hspace{1.8cm}$t$\hspace{0.9cm}$x_{av}$

\hspace{1.8cm} $\cdots$

\noindent donde $x_{av}$ es la coordenada longitudinal máxima que alcanza el agua en tiempo $t$.
 


\section{Formato del fichero de entrada de las sondas}
 
El formato del fichero de entrada se sondas es muy sencillo:

\begin{verbatimtab}
	(número de sondas)
	(coordenada x)
	...
\end{verbatimtab}

\section{Formato del fichero de resultados de las sondas}
El número de columnas de este fichero depende del número de sondas que hayan sido definidas en el fichero de entrada de sondas. 
En cada paso temporal guarda para cada sonda la pareja de valores ($h$, $s$):

\hspace{1.8cm}$t$\hspace{0.9cm}$h_1$\hspace{0.9cm}$s_1$\hspace{0.9cm}$\cdots$\hspace{0.9cm}$h_N$\hspace{0.9cm}$s_N$

\hspace{3.8cm} $\cdots \cdots$

\noindent donde $N$ es el número total de sondas. 

\clearpage
\section*{Notación}

\begin{description}
\item $\alpha$ = área infiltrada,
\item $A$ = área mojada de la sección transversal superficial,
\item $a$ = es el coeficiente de infiltración de Kostiakov,
\item $D$ = distancia entre surcos,
\item $D_x$ = coeficiente adimensional de difusión superficial,
\item $g$ = constante gravitatoria,
\item $h$ = altura de agua superficial,
\item $i_c$ = velocidad de infiltración en suelo saturado,
\item $K$ = parámetro de infiltración de Kostiakov, 
\item $K_x$ = coeficiente de difusión superficial,
\item $N$ = número total de sondas, 
\item $n$ = número de Gauckler Manning,
\item $P$ = perímetro mojado de la sección transversal superficial,
\item $Q$ = caudal superficial,
\item $S_f$ = pendiente de fricción,
\item $s$ = concentración de soluto promedio en la sección transversal superficial,
\item $s_i$ = concentración de soluto infiltrada,
\item $t$ = coordenada temporal,
\item $u=\frac{Q}{A}$ = velocidad promedio en la sección transversal superficial,
\item $x$ = coordenada longitudinal,
\item $x_{av}$ = coordenada longitudinal máxima que alcanza el agua en tiempo $t$,
\item $z_b$ = coordenada vertical del fondo de la sección transversal,
\end{description}

\clearpage
\begin{thebibliography}{}

\bibitem[Burguete et~al., 2009]{JaviSurcos1}
Burguete, J., Zapata, N., Garc\'{\i}a-Navarro, P., Ma\protect{\"{\i}}kaka, M.,
  Play\'an, E., and Murillo, J. (2009).
\newblock Fertigation in furrows and level furrow systems. \protect{I}: Model
  description and numerical tests.
\newblock {\em ASCE Journal of Irrigation and Drainage Engineering},
  135(4):401--412.

\end{thebibliography}

\end{document}


